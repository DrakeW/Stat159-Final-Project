\section{Results}

To understand the data, we computed descriptive statistics and summaries of all variables and generated corresponding plots in EDA phase, which can be found in \tttext{data/eda-output.txt} and \tttext{images}. Since we are interested in studying the factors that affect diversity or graduation rates, we also obtained matrix of correlations. Please refer to eda-output.txt in the data section of this report, as the matrix is large. 

We regressed diversity scores and graduation rates on different predictors separately. For diversity model, after fitting all models to the full data sets, we summarized coefficients and test MSE values. While the coefficients vary across models, we can spot some trends shared in common. The predictor that had the largest effect on diversity score was level of institution \tttext{ICLEVEL}, which conveys the highest level of award offered at the institution: 4-year, 2-year, or less-than-2-year. There are several other less influential elements that identify the degree profile of the institution including highest degree \tttext{HIGHDEG} and predominant undergraduate degree \tttext{PREDDEG}. Percent of undergraduates receiving federal Loans \tttext{PCTFLOAN} and governance structure \tttext{CONTROL} (public/private nonprofit/private for-profit) are also identified as related predictors by all models except Lasso. In contrast, the predictors \tttext{DEBT_MDN} and \tttext{MN_EARN_WNE_P10}, which stand for cumulative median debt and mean earnings, barely affect diversity of colleges.

